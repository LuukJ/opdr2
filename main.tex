\documentclass{article}

\usepackage[utf8]{inputenc}
\usepackage{listings}
\usepackage{microtype}
\usepackage[english,dutch]{babel}
\usepackage[a4paper]{geometry}

\frenchspacing
\parindent=0pt
\setlength{\textheight}{25.7cm}
\setlength{\textwidth}{16cm}
\topmargin 260mm \advance \topmargin -\textheight
\divide \topmargin by 2 \advance \topmargin -1in
\headheight 0pt \headsep 0pt \leftmargin 210mm \advance
\leftmargin -\textwidth
\divide \leftmargin by 2 \advance \leftmargin -1in
\oddsidemargin \leftmargin \evensidemargin \leftmargin


\lstset{language=Python, showstringspaces=false, basicstyle=\small,
  numbers=left, numberstyle=\tiny, numberfirstline=false, breaklines=true,
  stepnumber=1, tabsize=4, 
  commentstyle=\ttfamily, identifierstyle=\ttfamily,
  stringstyle=\itshape, }
\title{Tweede programmeeropgave: DeCode}
\author{Luuk\\2260018\and Levi van Es\\2115409 }
\date{\today}

\begin{document}

\maketitle

\section{Code}

    \subsection{Coderen}
       De codeerfunctie itereert over elk character in "Text".
			 Hierbij gaat het bij elke iteratie over de controle van een
			 aantal condities.
			 De eerste conditie word gecheckt door een if statement waarbij
			 het huidige charakter word gecheckt of het een getal is.
			 Wanneer dit zo is wordt dit getal
			 opgeteld bij de string "getal".
			 Wanneer het complete getal is opgesteld wordt er 1 opgeteld bij
			 de juiste waarde in een lijst met hoevaak getallen in bepaalde bereiken
			 voorkomen, op hier later een histogram van te maken.
       
       
       \begin{lstlisting}[frame=single, language=python]  % Start your code-block
        
        input encode(text)
        
       "Who knew little charlie could run that fast. He kept on running faster fasteer and fasteeeeeerr !"
        
        output encode(text) 
        
        "Who knew lit2le charlie could run that fast. He kept on run2ing faster faste2r and faste6r2 !"
 

        \end{lstlisting}

       
       De volgende conditie kijkt of het huidige karakter hetzelfde is als de vorige karakter en op basis hiervan word er bepaald of het bij de teller "n" moet worden opgeteld. Is het karakter niet hetzelfde als het vorige karakter dan wordt het getal van de teller samen met het vorige karakter afgedrukt en begint de teller opnieuw.
       \newline
       
       Op het laatste word er gekeken naar een speciale encode voorwaarde
			 waarbij een aantal karakters op een speciale manier gedecodeerd moeten worden.
			 Hierbij gaat het over de charakters \texttt{"\textbackslash0123456789"}.
			 Deze karakters moeten gedecodeerd worden met een "\textbackslash" voor zich.
			 Het doel hiervan is het herkennen van deze speciale getallen tijdens het
			 decoderen en het voorkomen van ongewenste artifacten.
			 Hierbij zou bijvoorbeeld een \texttt{"\textbackslash n"}
			 gezien kunnen worden door python als
			 een line-break i.p.v een normaal stukje text. Deze manier van encoderen
			 zorgt ervoor dat dit niet gebeurd.
       
       
       \begin{lstlisting}[frame=single, language=python]  % Start your code-block
            
            input encode():
                "2800 3400, 2900, 2000, 4000, 4378,  8000, 9000, 6000, 5689, 3478, 4085, 7095, 1010,1785, 3000,8090
                : // \737:;/\]["
    
            output encode():
                \2\8\02 \3\4\02, \2\9\02, \2\03, \4\03, \4\3\7\8, 2\8\03, \9\03, \6\03, \5\6\8\9, \3\4\7\8, \4\0\8\5, \7\0\9\5, \1\0\1\0,\1\7\8\5, \3\03,\8\0\9\0
                : /2 \\\7\3\7:;/\\][
        \end{lstlisting}
       
    \subsection{Decoderen}
    De ontwikkeling van het decoderen is opgebouwd uit verschillende delen. Het Decompressiegedeelte, de escapevoorwaarde en de palindroombepaling. 
    
     \subsubsection{Decompressie}
    Het compressiegedeelte werkt op basis van getallen die de repetitie van
		het vorige getal bepalen. Zo word er in de code de text karakter voor
		karakter uitgelezen waarbij bij elke iteratie ook het vorige karakter
		onthouden word. Wanneer de code een getal tegenkomt na een karakter voert
		het dit vorige getal keer het gelezen karakter uit naar het uitvoerbestand. 
    
    \subsubsection{Palindroom}
    De palindroombepaling werkt door het getal waar de palindroombepaling op
		uitgevoerd moet worden op te slaan in een accumulator.
		Wanneer het programma het keyword ``:P'' tegenkomt word de functie
		palindroom(accu) aangeroepen met de huidige accumulator als parameter.
		Deze accumulator word voor elke iteratie van de functie vergeleken met
		zijn eigen inverse. Wanneer het bij een iteratie niet resulteerd in een
		palindroom word het laatste getal van de accumulator afgehaald en word
		de palindroom-check nog een keer gedaan. Dit word herhaald tot de functie
		een palindroom tegenkomt.
    \begin{lstlisting}[frame=single, language=python]
    def palindroom(accu):
    """ retourneert (accu, reductions)
    
      accu - de accumulator van het decoderen
    
    Verwijdert het laatste cijfer tot het getal een palindroomgetal is
    en houdt daarbij in `reductions` bij hoevaak dit gebeurt.
        """
        string = str(accu)
        reductions = 0
        while string != string[::-1]:
            reductions += 1
            string = string[:-1]
    return accu, reductions
    
    \end{lstlisting}
        
    \subsubsection{Escapevoorwaarde}
    De escapevoorwaarde in onze Decodeercode heeft de functie om bepaalde stukken van de functie over te slaan wanneer de situatie bepaald is. Zo word de rest van de funcite niet uitgevoerd wanneer er een // word gededecteerd en wanneer er een palindroom of accumulator reset uitgevoerd moet worden. Dit zorgt ervoor dat er geen onnodige code uitgevoerd moet worden en het programma zo efficient loopt. Deze escape voorwaarde zorgt ervoor dat wanneer deze word gededecteerd, het het volgende karakter letterlijk moet opvatten en hier geen speciale operatie op moet toepassen. Op deze manier word het in de code ook mogelijk om backslashes te coderen.
    
    \subsection{Histogram}
      De ontwikkeling van het histogram ging relatief rechtoe rechtaan aangezien we beide een duidelijk beeld hadden van de aanpak. Het eerste prototype was schetsend en zeer tijdrovend. Hoewel dit "schetsje" zeer overzichtelijk was was het ook heel onnodig groot gecodeerd. Dat was voor ons de reden om een compactere versie van het histogram te maken. De schets maakte gebruik van een systeem waarbij het historgram al vooraf ruimtelijk was getekend en geinitaliseerd moest worden met standaard variabelen. Na deze initiatie werden de correcte coordinaten(variabelen) in het histogram vervangen door sterretjes. 
      \begin{lstlisting}[frame=single, language=python] 
    def histogram(bins):
    
        rows = [["*" if bins[x]/max(bins)*10 > (y+0) else " " for x in range(10)]
                for y in range(10)]
        out = str(max(bins))
        for row in reversed(rows):
            out += "\n |" + " ".join(row)
        out += "\n +" + "-"*20
        out += "\n1" + " "*20 + "9999"
    return out
      \end{lstlisting}
      
      \newpage
      \begin{lstlisting}[frame=single, language=python]  % Start your code-block
            
            input encode():
                "2800 3400, 2900, 2000, 4000, 4378,  8000, 9000, 6000, 5689, 3478, 4085, 7095, 1010,1785, 3000,8090
                : // \737:;/\]["
    
            output encode():
               Regels verwerkt: 2
                3
                 |    * * *          
                 |    * * *          
                 |    * * *          
                 |  * * * *       *  
                 |  * * * *       *  
                 |  * * * *       *  
                 |* * * * * * * * * *
                 |* * * * * * * * * *
                 |* * * * * * * * * *
                 |* * * * * * * * * *
                 +--------------------
                1                    9999
        \end{lstlisting}
      
      De uiteindelijke gecomprimeerde versie van de histogramfunctie maakt gebruik van een for-loop waarbij het tekenen en initialisatie van het programma in een loop verwerkt zitten. Dit stukje bespaart ons ongeveer 40 regels in het uiteindelijke product en is veel efficienter voor het werkgeheugen omdat het hierbij geen onnodige varabelen op moet slaan. 
    

    
    \subsection{Verhouding}
        De verhouding word in ons programma berekend door eerst het invoerbestand lijn voor lijn om te zetten naar het UTF-8 formaat zodat wij zeker weten dat elk karakter dezelfde ruimte van 8 bits inneemt. Hierdoor neemt elk karakter dezelfde ruimte in waardoor het makkelijk is om mee te rekenen. Hierbij word er ook rekening gehouden met veel speciale karakters in unicode die bestaan uit meer dan 1 byte. De verhouding word gerekend door de  naar UTF-8 omgezette lijn voor het encoderen te delen door de naar UTF-8 omgezette lijn na het encoderen. Dit resulteerd in het compressieratio van de gecomprimeerde textfile. 
        \newpage
        \begin{lstlisting}[frame=single, language=python]  % Start your code-block
        input():
            "Who knew little charlie could run that fast. He kept on running faster fasteer and fasteeeeeerr !"
            
            
        
        
        Output encode():    
            
            "Who knew lit2le charlie could run that fast. He kept on run2ing faster faste2r and faste6r2 !"
                
            Compressierate: 95%
            Regels verwerkt: 1

            
        \end{lstlisting}

\section{Tijdsverdeling}

\begin{tabular}{ l | p{6cm} p{6cm} }
   & Luuk & Levi \\
  \hline
  Week 1 & Encodeer gedeelte van de functie uitdenken en uitwereken en debuggen & Encodeergedeelte van de functie uitdenken en debuggen \\
  Week 2 & Een begin maken aan Decodeerfunctie en ideen opdoen voor palindroomuitwerking & Ontwerpen en uitwerken van eerste Histogramcode prototype \\
  Week 3 & -Ideeen opdoen voor verdere decodeeroptimalisatie en aanpassingen maken aan de decodeerfunctie om de code zo efficient en klein mogelijk te maken.\linebreak -Integratie van palindroomfunctie in de decodeerfunctie & -Verdere uitwerk van Histogramcode en gezamelijke Decoderen Debugging. \linebreak - Ontwikkeling uiteindelijk geimplementeerde versie van het histogram.  \linebreak  - Onderzoek naar python classes \\
  Week 4  & Debuggen van speciale gevallen bij het en- en decoderen. Implementeren van het compressieratio & Debugging an problemen die optraden tijdens het runnen van testbestanden \\
 
 
 
  
\end{tabular}

\section{De gehele code}
\lstinputlisting{2115409-2260018-opdr2.py}

\end{document}
